%\documentclass[a4paper,12pt]{report}
%\addtolength{\textwidth}{2cm}
%\addtolength{\topmargin}{-2cm}
%\addtolength{\textheight}{3.5cm}
%\newcommand{\HRule}{\rule{\linewidth}{0.5mm}}

%\begin{document}
\section{Relevance of database systems in COS 326 and business organisations.}
\par {The IPv4/IPv6 network management model makes use of an Object-Oriented Database (OODB) which was taught in this module, COS 326. The concept of an object oriented database, is that it stores objects instead of data. It encapsulates combinations of data structures together with associated functions. Because of this, arbitrary data types can be stored within the "database." The OODB makes use of a management system (OODBMS) that provides data integration, overall control, and DBMS support facilities for all types of objects. Additionally, it emphasises the necessary characteristics to support large, shared, persistent object stores; which include efficient processing over large secondary storage organisations, concurrency control, recovery facilities, and efficient processing of set-oriented requests also known as queries.} \cite{a}
%\end{document}