%\documentclass[a4paper,12pt]{report}
%\addtolength{\textwidth}{2cm}
%\addtolength{\topmargin}{-2cm}
%\addtolength{\textheight}{3.5cm}
%\newcommand{\HRule}{\rule{\linewidth}{0.5mm}}

%\begin{document}
\section{Solution to IPv4/IPv6 Trasition}

\subsection{Proposed Management Model}
There is a proposed database management model to cope with the coexistence of IPv4 and IPv6. This entails a distributed object-oriented database and hierarchical network management architecture. It mainly comprises of three particular components, namely, the Monitor, Collector and Distributed Object-
Oriented Database. These basically provide an insfrastructure for different IPv4/IPv6 configurations and abstracts the dependence on the underlying architecture. Also incorporated into the this management model is the REST architecture and the Perspective Broker protocol. \cite{Zhao}

\subsubsection{Monitor}
Basically, the Monitor simultaneously listens on IPv4 and IPv6 connections. Therefore facilitating effective management of purely IPv4, IPv6 and dual stacks without any extensive overhead. This is achieved through a number of components that, together, compose the Monitor. These are the modeling module, performance
collecting module, SNMP/SSH channel module, events module and user interface. The user interface. The monitor can be found below the Distributed Object-Oriented Database and above the Collector on the proposed hierarchy of componenets that form the management model. It is therefore effectively a link between the two.
%\end{document}